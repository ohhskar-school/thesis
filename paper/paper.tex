\documentclass{report}

% Page Size
\usepackage[letterpaper, portrait, margin=1in, left=1.5in]{geometry}

% Spacing
\usepackage{setspace}
\setlength{\parskip}{\baselineskip}
\doublespacing{}


% Bibliography
\usepackage[american]{babel}
\usepackage[style=apa, citestyle=apa, backend=biber]{biblatex}
\DeclareLanguageMapping{american}{american-apa}
\addbibresource{references.bib}
\usepackage[babel,threshold=2]{csquotes}

% Graphics
\usepackage{graphicx}
\graphicspath{ {./images/} }

% Table of Content
\setcounter{tocdepth}{2}

% Code Formatting
\usepackage{minted}
\usemintedstyle{pastie}

\def\code#1{\texttt{#1}}

% Positioning
\usepackage{float}

% Colors
\usepackage[dvipsnames]{xcolor}

% Links
\usepackage{hyperref}
\hypersetup{
	pdftitle={Development of a touch typing trainer with an emphasis on finger and wrist placement},
	pdfstartview={FitH},
	colorlinks=true,
	linkcolor=black,
	citecolor=black,
	filecolor=black,
	urlcolor=MidnightBlue
}


\begin{document}

\pagenumbering{gobble}
\begin{titlepage}
	\centering
	\includegraphics[width=0.25\textwidth]{upc.png}
	\includegraphics[width=0.25\textwidth]{dcs.png}
	\par\vspace{1cm}
	{\scshape\LARGE Research Brief \par}
	\par\vspace{1cm}
	\begin{tabular}{ | l | l | }
		\hline
		Name               & Oscar Vian L. Valles                  \\
		\hline
		Student ID number  & 2018--06491                           \\
		\hline
		Project Title      & Development of a touch typing trainer \\ & with an emphasis on finger and wrist positions\\
		\hline
		Project No.        & 9                                     \\
		\hline
		Name of Supervisor & Dhong Fhel K. Gom-os                  \\
		\hline
	\end{tabular}




\end{titlepage}
\newpage

\tableofcontents
\newpage

\pagenumbering{arabic}

\setcounter{chapter}{1}
\section{Background of the Study}
There are a lot of educational typing tests available that help people learn
touch typing, including Monkeytype, TypeRacer, or Keybr. These typing tests list
out words that are then typed out. The inputted keys are then compared to check
if the user has typed the expected letter. At the end of the test, the time
taken is calculated and certain metrics is given. These metrics include words
per minute (WPM) and accuracy \parencite{bartnik2021}.

However, this method of examination leaves out a crucial part of typing —
ergonomics. Ergonomic typing prevents a lot health issues in the future like
repetitive strain injury or carpal tunnel. One important factor that affects
ergonomics is the typing procedure and posture. This means proper placement of
the wrist, hands, and hitting the keys using the right finger that is assigned
to the key.

Correct finger placement is usually taught at the beginning using a diagram,
with each key being associated with a specific finger. For instance, the letter
Q in a QWERTY layout should be hit using the fifth digit of the left hand, and
this is shown by coloring the fifth digit and the key Q with the same color or
by placing the letters directly on the fingers \parencite{dobson2009touch}.

Incorrect finger placement may cause these hand and wrist positions: ulnar
deviation, forearm pronation, and wrist extension \parencite{serina1999}. These
three are hand and wrist positions that are common in all activities, however,
prolonged periods in these positions may cause cause injuries such as Carpal
tunnel syndrome (CTS) \parencite{toosi2015}

In addition, this type of typing is frequently taught in the beginner level
\parencite{donica2018}. This means that there is a need to weed out bad habits
that may develop, like using the index finger for pressing the spacebar or
backspace. However, it is impractical for an educator to check each individual
student if they are not performing these movements as these may only show for a
specific which may not be common.

Thus, there is a need for automatically detecting which finger is used during
typing, and for the position of the wrist in relation to the arm. One way to do
this is through finger and hand tracking. One solution for tracking is by using
image processing and machine learning. An example of this is MediaPipe by
\cite{mediapipe}.

MediaPipe allows for various applications for machine learning in the field of
image processing. This includes, hand tracking, pose estimation, object
detection, and others. Another example of a library that allows for hand and
finger tracking is OpenCV by \cite{opencv}. This is a tool that simplifies for
computer vision, and image processing. Machine learning can also be used with
OpenCV.

This research aims to improve touch typing training by combining hand and finger
tracking, with educational typing tests to determine if the finger used to type
is correct and if the overall posture of the hand ergonomic and healthy.

\section{Significance of the Research}
This research is beneficial for all users of physical keyboards. These include a
vast majority of the population as there are a lot of professions that heavily
rely on keyboards. Examples include developers, physicians, educators,
accountants. By having better ergonomics while typing, wrist injuries can be
prevented, and typing speed may be increased

This research also helps educators, especially early educators teaching beginner
typers. By automatically checking for ergonomics and posture, the burden of
checking each individual students is lessened, and directed interventions for
bad habits can be easily created as students with these bad habits are easily
identified

This research has a direct impact on people that has onset RSI or other
hand/wrist injuries that are caused by poor typing habits. By correcting these
poor habits, pain from these injuries will be lessened, and even be prevented
from occurring in the first place. A specific example of this is by reducing
ulnar deviation which affects the nerve that is indicative of CTS
\parencite{toosi2015}.

\newpage
\nocite{*} \printbibliography[heading=bibintoc,title={References}]{}

\end{document}
